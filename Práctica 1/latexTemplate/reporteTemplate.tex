%%
%% This is file `sample-sigconf.tex',
%% generated with the docstrip utility.
%%
%% The original source files were:
%%
%% samples.dtx  (with options: `sigconf')
%% 
%% IMPORTANT NOTICE:
%% 
%% For the copyright see the source file.
%% 
%% Any modified versions of this file must be renamed
%% with new filenames distinct from sample-sigconf.tex.
%% 
%% For distribution of the original source see the terms
%% for copying and modification in the file samples.dtx.
%% 
%% This generated file may be distributed as long as the
%% original source files, as listed above, are part of the
%% same distribution. (The sources need not necessarily be
%% in the same archive or directory.)
%%
%% The first command in your LaTeX source must be the \documentclass command.
\documentclass[sigconf]{acmart}

\usepackage[T1]{fontenc}
%%
%% \BibTeX command to typeset BibTeX logo in the docs
\AtBeginDocument{%
	\providecommand\BibTeX{{%
\normalfontB\kern-0.5em{\scshape i\kern-0.25em b}\kern-0.8em\TeX}}}

%%
%% end of the preamble, start of the body of the document source.
\begin{document}

%%
%% The "title" command has an optional parameter,
%% allowing the author to define a "short title" to be used in page headers.
\title{Práctica no. 1}

%%
%% The "author" command and its associated commands are used to define
%% the authors and their affiliations.
%% Of note is the shared affiliation of the first two authors, and the
%% "authornote" and "authornotemark" commands
%% used to denote shared contribution to the research.
\author{Jair Antonio Bautista Loranca}
\email{a01365850@itesm.mx}
\affiliation{%
	\institution{Tecnológico de Monterrey} 
	\city{Monterrey, N.L.}
	\country{México}
}

\author{Max Zambada}
\email{}
\affiliation{%
	\institution{Tecnológico de Monterrey} 
	\city{Monterrey, N.L.}
	\country{México}
}

%%
%% The abstract is a short summary of the work to be presented in the
%% article.
\begin{abstract}
	Aquí va el abstract de la práctica
\end{abstract}

%%
%% This command processes the author and affiliation and title
%% information and builds the first part of the formatted document.
\maketitle

\section{Introducción}
Texto introductorio al tema en que se enfoca la práctica y lo que se 
desarrollará en ella. Se debe escribir un texto que introduzca el tema de la 
práctica, definición del problema, los objetivos, motivación,y resultados 
esperados\cite{Cohen07}.

\section{Conceptos previos}
Escribir conceptos teóricos empleados en el desarrollo de la práctica (fórmulas 
matemáticas, por ejemplo). Es un tipo de sección con todos los conceptos 
teóricos empleados.

\section{Metodología}
Describir la metodología (todos los pasos) a emplear para desarrollar la 
práctica. 

\section{Resultados}
Describir con detalle todos los resultados. Hacer una discusión de lo obtenido.
Se debe mostrar un enfoque analítico sobre los resultados generados. Mostrar 
absolutamente todos los productos obtenidos debido a los pasos mostrados en la 
metodología.

\section{Conclusiones y reflexiones}
Conclusiones generales de la práctica. Añadir una reflexión analítica por cada 
miembro del equipo.

%%
%% The next two lines define the bibliography style to be used, and
%% the bibliography file.
\bibliographystyle{ACM-Reference-Format}
\bibliography{references}


\end{document}
\endinput
%%
%% End of file `sample-sigconf.tex'.
